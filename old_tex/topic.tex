\section{Topic}

 This project will focus on Evolving Artificial Neural Networks (EANN), a widely-explored alternative to back-propagation paradigm of network trainingThis project will focus on Evolving Artificial Neural Networks (EANN), a widely-explored alternative to back-propagation paradigm of network training \cite{DowningIntelligenceSystems}. While capable of generating recurrent structures and not requiring supervised learning during the training process, this approach can be stymied by its own set of challenges. In particular, designing these systems to be highly evolvable -- e.g., making them compatible with the evolutionary process by avoiding excessive fatal mutations and premature dead-end local maxima in the search space, for instance -- is a non-trivial task (especially for large networks). It is thought that moving beyond direct encodings where each topological connection and weight is explicitly specified, to adaptive, implicit, or generative genotype-phenotype mappings might address these issues. Promising research has been conducted in this vein. A prime example is the Genetic Regulatory Network (GRN) scheme of Reisinger and Miikkulainen, which they found yielded a more compact representation, more adaptive variation, and more robustness to mutation compared to the direct encoding (as well as alternate encodings) \cite{ReisingerAcquiringRepresentations}. This GRN scheme is just one of a number of genotype-phenotype mappings that have been explored, such as Neuroevolution of Augmenting Topologies (NEAT), cellular encoding (CE), and encodings based on context-free grammars \cite{DowningIntelligenceSystems}.
