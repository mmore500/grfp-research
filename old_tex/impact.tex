\section{Research Impact}

Given the interdisciplinary nature of EANN research, experimental results of this project will be of interest in many scholarly circles.
It is especially pertinent to evolutionary biology, which is grappling with the role of phenotypic plasticity in evolution theory (among several contentious issues) as grapples with the Extended Evolutionary Synthesis, a controversial raft of theory that proponents claim add vital explanatory power to the existing Modern Synthesis.\autocite{Pigliucci2008IsEvolvable, Laland2014DoesRethink}
Results describing evolutionary bias induced by representative and elitist models of phenotypic plasticity will directly inform these important ongoing conversations.

\section{Broader Impacts}

The techniques we develop to evolve HPFIR ANNs will also be useful outside academic research.
The possibility of evolving neural topologies that act as a springboard to boost deep learning, is particularly exciting; this research will contribute to the innovative products and services being unleashed by advances in artificial intelligence.

I aim to  concrete contributions to the scientific community, as well.
I will implement computational experiments using the Modular Agent Based Evolution Framework (MABE)to enable easy plug-and-play reuse by other researchers \autocite{Hintze2017Mabe}.
The techniques for evolving ANNs with high potential for irregular refinement developed through this project are naturally extensible to many problem domains, such as robot control, computer vision, and game-playing agents.
We plan to work with undergraduate researchers to demonstrate applications of such evolved networks and exploit the Empirical library to develop a web-based tool to evolve ANNs with high potential for irregular refinement and export those networks in formats compatible with existing deep learning frameworks, (including Caffe, TensorFlow, Theano, and Torch) for further refinement.
In our second, third, and fourth years, we will pursue grant support through the NSF BEACON Center to host a student through the MSU Summer Research Opportunities Program, which targets underrepresented undergraduate students.
