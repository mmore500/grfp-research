\section{Motivation}

Neural architectures with high potential for irregular refinement, which serve as a platform upon which techniques such as direct-encoded evolutionary search, online application of local learning rules, or backpropagation can operate successfully are highly desirable. However, directly selecting for irregular refinement potential by performing complete irregular refinement on all candidate solutions is computationally prohibitive. In the spirit of the Baldiwn effect, I propose that local phenotypic search via mutation to the direct representation of all candidate solutions could bias evolutionary search toward neural architectures with high potential for irregular refinement. While experimental results validate the Baldwin hypothesis\cite{Downing2009ComputationalEffect}, the type and degree of local phenotypic search necessary to bias evolutionary search to favor high irregular refinement potential is unknown. A better understanding of this question will allow for the evolution of neural architectures highly condusive to irregular refinement as well as furthering the theoretical conception of the interplay between phenotypic plasticity and evolution.