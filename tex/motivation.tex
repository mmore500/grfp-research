\section{Motivation}

Although HPFIR ANNs with are highly desirable, directly selecting for irregular refinement potential by performing backpropagation training or direct-encoded evolutionary search on all candidate solutions is computationally prohibitive.
I hypothesize that local phenotypic search via mutation to the direct representation of candidate solutions, in addition to supporting the evolution of heritable scaffolding for advantageous traits vis-\`{a}-vis the Baldwin effect, might select for characteristics that promote greater potential for irregular refinement via backpropagation graining or direct-encoded evolutionary search.
The proposed research will characterize evolutionary bias towards HPFIR ANNs induced by local phenotypic search, specifically considering (1) the organization of local phenotypic search, the degree to which that search is biased by distribution of fitness over the local phenotypic space, and (2) the selective interpretation of the results of that local phenotypic search, the degree to which the presence of fitness peaks is considered over the overall fitness of the local phenotypic neighborhood.
A better understanding of these questions will translate to methodology that promotes evolution of neural architectures highly conductive to irregular refinement through techniques such as backpropagation training as well as furthering the scientific understanding of the interplay between phenotypic plasticity and evolution.
