\section{Motivation}

While neural architectures with high potential for irregular refinement are highly desirable, directly selecting for irregular refinement potential by performing complete irregular refinement on all candidate solutions is computationally prohibitive.
In the spirit of the Baldiwn effect, I propose that local phenotypic search via mutation to the direct representation of candidate solutions could bias evolutionary search toward neural architectures with high potential for irregular refinement.
While experimental results validate the Baldwin hypothesis,\autocite{Downing2009ComputationalEffect} the type and degree of local phenotypic search necessary to meaningfully bias evolutionary search is unknown.
The proposed research will characterize evolutionary bias towards artificial neural architectures with high irregular refinement potential induced by varied methods of local phenotypic search, specifically considering (1) the organization of local phenotypic search, the degree to which that search is biased by distribution of fitness over the local phenotypic space, and (2) selective preference for characteristics of local phenotypic space, the degree to which the presence of fitness peaks is considered over the overall fitness of the local phenotypic neighborhood.
A better understanding of these questions will translate to methodology that promotes evolution of neural architectures highly conductive to irregular refinement as well as furthering the theoretical conception of the interplay between phenotypic plasticity and evolution.
 
