\section{Background}

Neuroevolution aims to design ANN topologies and weighting schemes through repeated evaluation, selection, and recombination of candidate solutions, an algorithm inspired by biological evolution.
In the biological metaphor, candidate solutions have a phenotype (an ANN configuration that evolutionary selection acts upon) and a genotype (information that recombination and mutation act on and from which the phenotype is constructed).

Direct genetic encodings, where each phenotypic trait is stored verbatim in the genome, allow fine-toothed exploration of the evolutionary search space.
Conversely, indirect genetic encodings, where genetic information is expanded via a process that allows for reuse of that information to uniformly describe many phenotypic characteristics, biases evolutionary search towards phenotypic regularity.\autocite{Clune2011OnRegularity}
As a result of this bias towards regularity, in highly regular problem domains indirect genetic encoding outperforms direct genetic encoding.\autocite{Clune2011OnRegularity}
Nevertheless, most problem domains exhibit some irregularity, which can often only be exploited by irregular phenotypic structures; experimental evidence indicates that champion solutions evolved with indirect genetic encodings be made more fit by irregular refinement vis-\`a-vis direct encoded evolutionary search.\autocite{Clune2011OnRegularity}

Learning, a type of phenotypic plasticity that provides another route to irregular refinement,\autocite{Clune2011OnRegularity} is thought to bias evolutionary search towards discovering useful adaptation through a mechanism known as the Baldwin effect.\autocite{Downing2010TheNetworks}
By allowing a candidate solution to assume proximal phenotypic forms, learning enables evolutionary selection to act on information about a candidate solution's local phenotypic neighborhood.
Through selective pressure for phenotypes proximal to high-fitness phenotypic forms, local phenotypic search ``buys evolutionary time'' until heritable scaffolding arises to support phenotypic adaptation originally attained via plasticity.\autocite{Downing2010TheNetworks}
