\section{Background}

Neuroevolution aims to design artificial neural network (ANN) topologies and weighting schemes through repeated evaluation, selection, and recombination of candidate solutions, an algorithm inspired by biological evolution. In the biological metaphor, candidate solutions have a phenotype, an ANN configuration that evolutionary selection acts upon, and a genotype, information that mutation and recombination act on and from which the phenotype is constructed. Genotypic representations are said to be either direct, indicating that each characteristic of the phenotype is stored verbatim in the genome, or indirect, indicating that the phenotype is generated from information that does not bear a one-to-one relationship to phenotypic characteristics. Direct mappings, which allow evolution to tweak individual phenotypic characteristics of candidate solutions independently, alllow fine-tooothed exploration of the phenotypic search space. Conversely, indirect genetic encodings, where genetic information is expanded via a process that allows for reuse of that information to uniformly describe a large number of phenotypic characteristics, has been demonstrated to bias evolutionary search towards phenotypic regularity\autocite{Clune2011OnRegularity}.

As a result of this bias, in highly regular problem domains evolutionary search with indirect genetic encoding outperforms search relying upon direct genetic encoding \autocite{Clune2011OnRegularity}. Nevertheless, most problem domains exhibit some irregularity, which often can only be exploited irregular phenotypic refinement; experimental evidence indicates that champion solutions evolved with indirect genetic encoding can often be bettered by irregular refinement vis-\`a-vis further evolutionary search with a direct encoding \autocite{Clune2011OnRegularity}. Clune et al. postulate that learning (post-developmental phenotypic plasticity of a neural network) similarly constitutes irregular refinement, enhancing the performance of a highly regular indirectly encoded phenotype via irregular modifications. Further, by allowing a candidate solution to assume proximal phenotypic forms, learning enables evolutionary selection to act on information about a candidate solution's local phenotypic neighborhood. Through selective pressure for phenotypes proximal to high-fitness phenotypic forms, local phenotypic search ``buys evolutionary time'' until heritable scaffolding arises to support phenotypic adaptation originally attained via plasticity \autocite{Downing2010TheNetworks}. This concept is termed the Baldwin effect.