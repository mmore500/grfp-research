\section{Impact}

Given the interdisciplinary nature of EANN research, experimental results of this project will be of interest in many scholarly circles.
It is especially pertinent to evolutionary biology, which is grappling with the role of phenotypic plasticity in evolution theory (among several contentious issues) as grapples with the Extended Evolutionary Synthesis, a controversial raft of theory that proponents claim add vital explanatory power to the existing Modern Synthesis.\autocite{Pigliucci2008IsEvolvable, Laland2014DoesRethink}
Results describing evolutionary bias induced by representative and elitist models of phenotypic plasticity will directly inform these important ongoing conversations.

The techniques we develop to evolve ANN with high irregular refinement potential will be useful both inside and outside academic research.
The possibility of evolving neural topologies that boost the performance of backpropagation training, the powerful and widely-used algorithm behind much of the recent progress in applied deep learning, is particularly exciting.
This research will contribute to the tangible, pervasive, and profoundly innovative products and services being unleashed by advances in artificial intelligence.

This project will yield immediate, concrete contributions to the scientific community, as well.
Computational experiments will be implemented using the Modular Agent Based Evolution Framework \autocite{Hintze2017Mabe}.
This way, software components will be made available under a MIT license for easy, plug-and-play reuse by other researchers.
The techniques for evolving ANN with high potential for irregular refinement developed through this project are naturally extensible to many problem domains, such as game-playing agents, robot control, and image classification.
We plan to work with undergraduate researchers to demonstrate applications of such evolved networks and exploit the Empirical library to develop a web-based tool to evolve ANN with high potential for irregular refinement and export those networks in formats compatible with existing deep learning frameworks, (including Caffe, TensorFlow, Theano, and Torch) for further refinement.
In our second, third, and fourth years, we will pursue grant support through the NSF BEACON Center to host a student through the MSU Summer Research Opportunities Program, which targgets underrepresented undergraduate students.
