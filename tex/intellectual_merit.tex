\section{Intellectual Merit}
By considering irregular refinement as a model of learning, this research brings in a new toolkit --- based on the work of Clune et al. \cite{Clune2011OnRegularity} --- to more precisely describe the Baldwin effect. This represents a novel link between to previously separate lines of ANN research \cite{Clune2011OnRegularity,Downing2010TheNetworks}. By comparing the effectiveness of
\begin{itemize}
  \item naive to directed local phenotypic search, and
  \item considering median versus best phenotypic fitness discovered during local phenotypic search,
\end{itemize}
at biasing evolutionary search towards networks with potential for enhancement via irregular refinement this research has the potential to inform current efforts to  understand what characteristics make neural architectures amenable to learning and to develop methedology to discover those networks via evolutionary search.

The impact of this research will extend beyond the field of Bio-AI to the disciplines that EANN research draws heavily from: neuroscience, evolutionary biology. Because of the interdisciplinary nature of EANN research, experimental results and theoretical advances from EANN are of interest to a huge community of researchers; EANN research offers significant potential to cross-pollinate ideas between these fields \cite{Pigliucci2008IsEvolvable, MoczekTheInnovation, MouretImportingGanglia}. Experimental results of this project, in particular, will be pertinent to evolutionary biology, which is grappling with the role of phenotypic plasticity in evolution theory --- among other a handful of other contentious questions --- as it undergoes a paradigm shift in evolution theory from the Modern Synthesis to the Extended Evolutionary Synthesis (EES) \cite{Pigliucci2008IsEvolvable}. Results describing the evolutionary bias induced by models of phenotypic plasticity --- differ in search naivety, maximum phenotypic distance traversed  during search, and in treatment of fitness as the maximum or median fitness of explored phenotypes --- has the potential to directly inform these important conversations in the evolutionary biology. 

% \begin{itemize}
%   \item implications to broad questions of learning in EANN, the ultimate goal is to develop plastic EANN capable of on-line modification
%   \item 
%   \item interdisciplinary: bridge gap between evolutionary biology and computer science
%   \item advances in EANN can bridge back to inform/expand our understanding of the evolutionary process \cite{MoczekTheInnovation} \cite{Pigliucci2008IsEvolvable} and emergent intelligence (i.e. neuroscience) in vivo
%   \item ``Extended Evolutionary Synthesis'' ``The broader context is that evolvability will constitute one of the foundational blocks for the much anticipated (or dreaded) EES in evolutionary biology93, together with other concepts that are new to — and yet build upon the achievements of — the Modern Synthesis. This expan- sion will include the role of phenotypic and behavioural plasticity 4,7, a better understanding of the evolution of develop- ment8–13, the role of epigenetic inheritance systems94, the idea of genetic accommoda- tion7, the dynamics of evolution in highly dimensional adaptive landscapes57, and of course the wealth of information provided by the post-genomic era95. It is an exciting moment to be an evolutionary biologist.'' \cite{Pigliucci2008IsEvolvable}
%   \item The Baldwin effect postulates that network plasticity can pave the way for heritable variation \cite{Downing2009ComputationalEffect}
%   \item Baldwin effect and learning: the most naive version; how much sophistication is required in the learning process to observe this effect
% \end{itemize}