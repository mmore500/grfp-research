\renewcommand{\thefootnote}{\fnsymbol{footnote}}
\section{Research Plan}
A variant of the HybridID \autocite{Clune2011OnRegularity} algorithm will be employed; evolutionary search will be run with on an indirect genetic encoding (HyperNEAT) then continued with a direct encoding (FT-NEAT), both for a set number of generations. Phenotypic plasticity will be realized (exclusively during the indirectly-encoded evolutionary stage) by generating 200 phenotypic variants at each evaluation of a candidate solution through simulated-annealing local phenotypic search via mutation of direct-encoded representations of that candidate solution.\footnote[1]{While generating and testing a large number of phenotypic variants at each evaluation is computationally expensive, this approach is highly parallelizable and feasible with the resources available to academic research groups. \autocite{Mengistu2016EvolvabilityIt}} Experiments will be replicated at varying levels of problem regularity across three domains (each designed to permit ready manipulation of problem regularity): target weights, bit mirroring, and quadruped control\autocite{Clune2011OnRegularity}. 

In the control trial, evolutionary selection will be performed according to raw individual fitness. In the representative plasticity trial, phenotypic variants will be generated via slow-cooling simulated annealing search and selection will be performed according to median fitness of phenotypic variants encountered. In the elitist plasticity trial, phenotypic variants will be generated via fast-cooling simulated annealing search and selection will be performed according to the maximum fitness of phenotypic variants encountered. A third trial will be conducted to test intermediate a simulated annealing cooling rate and fitness reporting scheme. For each trial, evolutionary bias will be assessed by tabulating champion solution performance at the end of the indirectly-encoded evolutionary stage, champion solution performance after the irregularly enhancing directly-encoded evolutionary stage, and the performance gained during the directly-encoded evolutionary stage.