\renewcommand{\thefootnote}{\fnsymbol{footnote}}
\section{Research Plan}
This project will employ a variant of the HybridID \autocite{Clune2011OnRegularity} algorithm;
evolutionary search will begin with an indirect genetic encoding (HyperNEAT) then continued with a direct encoding (FT-NEAT), both for a set number of generations.
Phenotypic plasticity will be realized during the indirectly encoded evolutionary stage by generating 200 phenotypic variants at each evaluation of a candidate solution through simulated annealing local phenotypic search via mutation of directly encoded representations of that candidate solution.
This approach is highly parallelizable and feasible with the resources available to academic research groups.\autocite{Mengistu2016EvolvabilityIt}
Experiments will be replicated at varying levels of problem regularity across three domains (each designed to permit manipulation of problem regularity): target weights, bit mirroring, and quadruped control.\autocite{Clune2011OnRegularity}

In the control trial, evolutionary selection will be performed according to raw individual fitness.
In the representative plasticity trial, phenotypic variants will be generated by slow-cooling simulated annealing search;
selection will be performed according to median fitness of phenotypic variants encountered.
In the elitist plasticity trial, phenotypic variants will be generated by fast-cooling simulated annealing search;
selection will be performed according to the maximum fitness of phenotypic variants encountered.
For each trial, evolutionary bias will be assessed by tabulating champion solution performance at the end of the indirectly encoded evolutionary stage and the performance gained during the directly encoded evolutionary stage.
Specifically, large performance gains during the second evolutionary stage relative to the control trial would indicate promotion of irregular refinement potential.
